Abstract of Bachlor's Thesis

\begin{flushright}
	Academic Year 2017
\end{flushright}

\begin{center}
	\begin{large}
		Inferring parking block from vehicle position data 
	\end{large}
\end{center}

The traffic congestion caused by searching for an empty parking lot has become a serious social problem especially in urban areas of Japan and major tourist spots.

Also, the congestion frequently occurs due to vehicles attempting to check the availability of a large parking area where you can not see the entire parking lot by visual inspection.
Some of the stores have introduced Central management type fullness management system but it is expensive so that everyone cannot afford it. 

On the other hand, an architecture that performs autonomous distributed control has been devised in recent years. This architecture does not necessarily require permanent equipment because it senses empty situation by sharing location information by Vehicle-to-Vehicle communication. However, it has the disadvantage of requiring the maintenance and operation of the parking area management entity because it is necessary to construct a network model of the entire parking area in advance.

Therefore, in this research, we propose a method for constructing the network model of the entire parking area in a specific area from the log data of the position information about vehicles in the parking area. By using this method, it is possible to predict the behavior of the vehicle in the parking lot and estimate the positional relationship of the parking section. Additionally, it is possible to predict the number of accommodated parking lots at full learning occasions from these data.

The verification experiment was carried out with Keio University Shonan Fujisawa campus parking area as a model case where there is a difference in the degree of the parking block of popularity and becomes vacant, and the loitering of the division search is easy to occur in order to evaluate this approach. We created a virtual map of a parking area imitating the campus and evaluated the generated network model on a proprietary traffic simulator. That proved the validity of the proposed algorithm.

By utilizing the proposed methods in this research, 
it becomes possible to grasp and visualize the vacancy situation of the entire parking area, and that leads to an optimized traffic volume. Even under similar environments such as urban areas and sightseeing spots where similar problems are likely to occur, there is an expectation to realize integrated visualization of vacancy situations beyond the framework of the parking management entity and efforts to reduce congestion.


Keywords :\\
\underline{1. ITS},
\underline{2. GPS},
\underline{3. Probe Vehicle Systems},
\underline{4. Mobile network},


%~ \bigskip ~ \\
\begin{flushright}
	Keio University, Faculty of Policy Management\\
	~ \\
	\begin{large}
		Yasunobu Toyota
	\end{large}
\end{flushright}

