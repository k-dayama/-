
卒業論文要旨 2017年度 (平成29年度)

~ \bigskip ~ \\
\begin{center}
\begin{large}
  自動車の位置情報ログデータを用いた\\
  駐車区画推定手法の提案
\end{large}
\end{center}
%## 背景$\cdot$問題提起

マイカー利用者による駐車場待ちや空き駐車場の探索により生じる交通渋滞は,日本の都市部や主要な観光地等において深刻な社会問題化している.

また,駐車場全体を目視で見渡すことの出来ない大型駐車場や階層型駐車場においても,施設への入り口や特定の店舗の近傍などの人気の高い区画に利用者が集中するため,空き状況の確認を試みる車両による渋滞が頻発する.一部の施設では超音波センサ等による常設筐体を用いて満空状況を管理する中央管理型システムが導入されているが,運営者が用意できるコストに依存している.
 
一方,近年では自律分散型制御を行うアーキテクチャも考案されている.これは利用車両に搭載された車々間通信端末によって位置情報を共有することで空き状況をセンシングするため,常設機器を必ずしも必要としない.しかし駐車場全体のネットワークモデルの事前の構築が必要になるため,結果として駐車場の運営主体による維持$\cdot$運営を必要とするという欠点がある.


そこで,本研究では駐車場内の車両の位置情報のログデータから,特定のエリアにおける駐車場全体のネットワークモデルを構築するための手法を提案する.本手法を用いることによって,駐車場内での車両の振る舞いを予測し,駐車区画の位置関係の推定が可能になる.加えて,それらのデータから学習満車時の各駐車区画の収容台数を予測することが可能となる.


この手法を評価するため,駐車区画の人気度に差があり空き区画探索の徘徊が発生しやすい慶應義塾大学湘南藤沢キャンパス内駐車場をモデルケースとし,検証実験を実施した.独自に実装した交通シミュレータ上でキャンパスを模した駐車場の仮想地図を作成し,駐車場内の各駐車区画の数や座標,駐車枠の保有台数に対して十分な推定精度を得るために必要な車両台数や時間に関する評価を通じて,本研究が提案するアルゴリズムの妥当性が証明された.

本研究が提案するネットワークモデル構築システムと既存研究で用いられた手法を活用することで,路上常設のセンサーを用いること無く,駐車場全体の空き状況の把握$\cdot$可視化$\cdot$最適な区画へのナビゲーションを行うことが可能になり,利用車両の無駄な徘徊の低減や駐車時間の減少を見込める.加えて都市部や観光地の様な同様の問題が発生しやすい環境下においても,駐車場運営主体の枠を超えた空き状況の統合的な可視化の実現と,渋滞の低減に対する取り組みがが期待される.

キーワード \\
\underline{1.ITS},
\underline{2.位置情報},
\underline{3.プローブ情報},
\underline{4. 携帯電話網}
\begin{flushright}
慶應義塾大学 総合政策学部\\
%~ \\
\begin{large}
  豊田 安信
\end{large}
\end{flushright}

