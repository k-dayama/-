
\chapter{結論}
\label{conclusion}
本章では本論文のまとめと今後解決すべき課題を示す.

\section{本研究のまとめ}

本論文では,始めに,路上に敷設した機器を用いる既存の中央管理型駐車場管理システムや,車々間通信による自律分散型管理システムに関する問題点を整理し,その代表として,ネットワークモデルの構築とシステムの管理が運営者の大きな負担であり導入障壁であると指摘した.

次に,既存の研究や手法から,自動車の位置情報を用いる駐車場の最適化手法の有用性を示した.

そこで本研究では,路上に設置する機器や運営主体による管理に依存せずに駐車区画の空き状況を収集$\cdot$可視化出来る環境を創出することを目的として,自動車の位置情報ログデータによって駐車場のネットワークモデルを自律的に推定する手法と,得られたネットワークモデルによって自律的に駐車場の満空状況を管理する新しいアーキテクチャを提案した.

また,駐車場のネットワークモデルを構成する要素を整理し,駐車区画の位置座標と保有する駐車枠数の推定を取り組むべき課題に設定した.本論文では,車両の駐車ログデータとX-Means法を用いたアプローチと,駐車区画の人気度の偏りに注目したアプローチによって自律的に推定する手法を提案した.

加えて,本提案手法の妥当性を検証するために,慶應義塾大学湘南藤沢キャンパスをモデルケースとしたシミュレータを実装し,2つの評価実験を行った.駐車区画の座標に関しては,駐車場の述べ200\%以上のログデータを収集$\cdot$解析することで,十分に信頼できる推定が行えることを示した.また,主要な駐車区画の駐車枠数に関しては,駐車場全体の満車率が70\%程度のログデータを用いることで,高い精度の推定が行えることを示した.

以上の実験の成果として,シミュレータ環境における本提案手法の妥当性が証明された.



\section{将来的な展望}
本論文では妥当性の検証方法として実地図をモデルにしたシミュレータ環境を用いたが,実環境下における有用性は再度検証する余地があると言える.

今後は,実際の車両の走行ログデータや,車載器の低普及環境下での検証を引き続き行い,本提案手法のフィジビリティを明確にしていくことが求められる.

本提案手法の推定精度を更に高めることで,実環境下においても路上常設のセンサーを用いること無く駐車場全体の空き状況の把握$\cdot$可視化$\cdot$最適な区画へのナビゲーションを行うことが可能になり,利用車両の無駄な徘徊の低減や駐車時間の減少を見込める.加えて,都市部や観光地の様な同様の問題が発生しやすい環境下においても,駐車場運営主体の枠を超えた空き状況の統合的な可視化の実現と,渋滞の低減に対する取り組みがが期待される.